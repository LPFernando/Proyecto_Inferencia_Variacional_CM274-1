\documentclass[twocolumn]{article}
%\documentclass{article}
% Margenes----------------------------------------------------------
\textheight = 24cm
\textwidth = 18.5cm % Ancho
\topmargin = -2.5cm
\oddsidemargin = -1cm
\parindent = 0mm % Sin sangría
%Paquetes adicionales-----------------------------------------------
%Otra opción para márgenes,etc., es el paquete geometry.
%\usepackage[total={18cm,21cm},top=2cm, left=2cm]{geometry}
\usepackage{latexsym,amsmath,amssymb,amsfonts}
\usepackage[latin1]{inputenc}
\usepackage[T1]{fontenc}
\usepackage{graphicx}
%\usepackage[lined,boxed]{algorithm2e}
%\usepackage[linesnumbered]{algorithm2e}
     
\inputencoding{utf8}
\usepackage[linesnumbered,ruled,vlined]{algorithm2e}
\usepackage{algorithmic}
\usepackage[spanish]{babel} % Idioma español 
\renewcommand{\baselinestretch}{1.0} % espaciado 1.1
\pagestyle{myheadings}
% \markright{...... texto .......}
\usepackage{enumerate}
\newcommand\tab[1][1cm]{\hspace*{#1}}
%-------------------------------------------------------------------
\begin{document} 

%titulo
\title{Universidad Nacional de Ingeniería\\
Introducción a la Estadística y Probabilidades\\
Inferencia Variacional}
\author{
Steve Anthony Luzquiños Agama\\
Aquí ponen sus nombres papus :u
\thanks{Aquí ponen un agradecimiento al profe Lara o quizás "primer avance" :u}
\\stevelaxd@gmail.com
}
\date{Aquí ponemos la fecha de presentación del primero avance}

\maketitle


%Seccion Introducción
\section{Introducción}
Supongo que aquí pondremos la introducción cuando la tengamos :u
%Fin del documento
\end{document}