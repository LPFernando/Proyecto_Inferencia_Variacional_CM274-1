\documentclass[a4paper,11pt]{article}
\usepackage[utf8]{inputenc}
\usepackage[spanish]{babel}
\usepackage{amsfonts}
\usepackage{amssymb}
\usepackage{amsmath}
\usepackage{multicol}
\usepackage[left = 3.00cm,right = 3.00cm,top = 3.00cm,bottom = 3.00cm]{geometry}
\usepackage{graphicx}
\usepackage[pdftex]{color}

\newtheorem{teor}{Teorema}[section]
\newtheorem{lema}{Lema}[section]
\newtheorem{defi}{Definici�n}[section]
\newtheorem{ejem}{Ejemplo}[section]
\newtheorem{propi}{Propiedad}[section]

\newcommand{\ds}{\displaystyle}

\begin{document}
	\begin{center}
		{\bf UNIVERSIDAD NACIONAL DE INGENIERIA\\
		FACULTAD DE CIENCIAS}
	\end{center}
	\begin{center}
		\includegraphics[scale=0.1]{uni.png}
	\end{center}		
	\begin{center}
		\rule{15cm}{0.8mm}\\
		\vspace*{3mm}
		{\bf ESTADISTICA VARIACIONAL\\
		REGRESION LINEAL VARIACIONAL\\
		REGRESION LOGISTICA VARIACIONAL}
		\rule{15cm}{0.8mm}\\
		\vspace*{4mm}
		{\bf Curso:}\\
		INTRODUCCION A LA PROBABILIDAD Y ESTADISTICA\\
		\vspace*{4mm}
		{\bf Integrantes:}\\
		\begin{flushleft}
			\hspace{5cm}$\bullet$ LOPEZ PUNIN RENEE JAIR\\
			\hspace{5cm}$\bullet$ LOAYZA FERNANDO\\
			\hspace{5cm}$\bullet$ JIMENEZ JOEL\\
			\hspace{5cm}$\bullet$ LUZQUINOS\\
		\end{flushleft}
		\vspace*{4mm}
		{\bf Profesor:}\\
		LARA AVILA CESAR\\
		\vspace*{6cm}
		LIMA-PERU\\
		CICLO 2018-II
	\end{center}
	\newpage
	{\bf INTRODUCCION}\\
	\textbf{Modelo de regresion lineal}\\
	El modelo de regresion lineal explica la relacion entre una variable dependiente, a la
	que se denotara y, con otra(s) explicativa(s) a traves de una ecuacion de primer orden.
	Tanto las variables dependientes como las explicativas son observables.
	
	\section{\bf ESTADO DEL ARTE}
	\subsection{Machine Learning}\\
	\subsubsection{Nacimiento(1952-1958)}\\
	1952 — Arthur Samuel escribe el primer programa de ordenador 
	capaz de aprender. El software era un programa que jugaba a 
	las damas y que mejoraba su juego partida tras partida.
	\\
	
	1956 — Martin Minsky y John McCarthy, con la ayuda de Claude Shannon
	y Nathan Rochester, organizan la conferencia de Darthmouth de 1956, considerada como el 
	evento donde nace el campo de la Inteligencia Artificial. Durante la conferencia, Minsky 
	convence a los asistentes para acuñar el término “Artificial Intelligence” como nombre 
	del nuevo campo.\\
	
	
	1958 — Frank Rosenblatt diseña el Perceptrón, la primera red neuronal artificial.
	
	\subsubsection{Primer Invierno(1974-1980)}\\
	En la segunda mitad de la década de los 70 el campo sufrió su 
	primer “Invierno”. Diferentes agencias que financian la 
	investigación en IA cortan los fondos tras numerosos años de 
	altas expectativas y muy pocos avances.\\
	
	
	1979 — Estudiantes de la Universidad de Stanford inventan el  
	“Stanford Cart”, un robot móvil capaz de moverse autónomamente
	por una habitación evitando obstáculos.
	
	\subsubsection{La explosion de los 80(1980-1987)}\\
	Los años 80 estuvieron marcados por el nacimiento de los sistemas expertos, basados en reglas. Estos fueron rápidamente adoptados
	en el sector corporativo, lo que generó un nuevo interés en Machine Learning.
	
	1985 — Terry Sejnowski inventa NetTalk, que aprende a pronunciar
	palabras de la misma manera que lo haría un niño.
	
	\subsubsection{Segundo Invierno(1987-1993)}\\
	A finales de los 80, y durante la primera mitad de los 90, llegó
	el segundo “Invierno” de la Inteligencia Artificial. Esta vez sus efectos se extendieron durante muchos años y la reputación 
	del campo no se recuperó del todo hasta entrados los 2000.\\
	1997 — El ordenador Deep Blue, de IBM vence al campeón mundial de ajedrez Gary Kaspárov.
	\subsubsection{Explosion y adopcion comercial(2006-actualidad)}\\
	El aumento de la potencia de cálculo junto con la gran abundancia
	de datos disponibles han vuelto a lanzar el campo de Machine Learning. Numerosas empresas están transformando sus negocios
	hacia el dato y están incorporando técnicas de Machine Learning en sus procesos, productos y servicios para obtener ventajas competitivas sobre la competencia.
	
	\subsection{Articulos Relevantes}\\
	
\end{document}