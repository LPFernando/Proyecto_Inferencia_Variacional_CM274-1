\documentclass[a4paper,11pt]{article}
\usepackage[utf8]{inputenc}
\usepackage[spanish]{babel}
\usepackage{amsfonts}
\usepackage{amssymb}
\usepackage{amsmath}
\usepackage{multicol}
\usepackage[left = 3.00cm,right = 3.00cm,top = 3.00cm,bottom = 3.00cm]{geometry}
\usepackage{graphicx}
\usepackage[pdftex]{color}

\newtheorem{teor}{Teorema}[section]
\newtheorem{lema}{Lema}[section]
\newtheorem{defi}{Definici�n}[section]
\newtheorem{ejem}{Ejemplo}[section]
\newtheorem{propi}{Propiedad}[section]

\newcommand{\ds}{\displaystyle}

\begin{document}
	\begin{center}
		{\bf UNIVERSIDAD NACIONAL DE INGENIERIA\\
		FACULTAD DE CIENCIAS}
	\end{center}
	\begin{center}
		\includegraphics[scale=0.1]{uni.png}
	\end{center}		
	\begin{center}
		\rule{15cm}{0.8mm}\\
		\vspace*{3mm}
		{\bf ESTADISTICA VARIACIONAL\\
		REGRESION LINEAL VARIACIONAL\\
		REGRESION LOGISTICA VARIACIONAL}
		\rule{15cm}{0.8mm}\\
		\vspace*{4mm}
		{\bf Curso:}\\
		INTRODUCCION A LA PROBABILIDAD Y ESTADISTICA\\
		\vspace*{4mm}
		{\bf Integrantes:}\\
		\begin{flushleft}
			\hspace{5cm}$\bullet$ LOPEZ PUNIN RENEE JAIR\\
			\hspace{5cm}$\bullet$ LOAYZA FERNANDO\\
			\hspace{5cm}$\bullet$ JIMENEZ JOEL\\
			\hspace{5cm}$\bullet$ LUZQUINOS\\
		\end{flushleft}
		\vspace*{4mm}
		{\bf Profesor:}\\
		LARA AVILA CESAR\\
		\vspace*{6cm}
		LIMA-PERU\\
		CICLO 2018-II
	\end{center}
	\newpage
	{\bf INTRODUCCION}\\
	\textbf{Modelo de regresion lineal}\\
	El modelo de regresion lineal explica la relacion entre una variable dependiente, a la
	que se denotara y, con otra(s) explicativa(s) a traves de una ecuacion de primer orden.
	Tanto las variables dependientes como las explicativas son observables.
\end{document}